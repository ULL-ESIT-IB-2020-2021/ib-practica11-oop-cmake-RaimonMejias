\subsubsection*{Objetivos}

Los objetivos de esta práctica son que el alumnado\+:
\begin{DoxyItemize}
\item Desarrolle programas sencillos en C++ que utilicen clases, así como todas las características del lenguaje estudiadas
\item Aloje todo el código fuente de sus programas en repositorios privados de Git\+Hub
\item Sepa depurar sus programas usando la interfaz de depuración del V\+SC
\item Incluya en sus programas comentarios en el formato requerido por Doxygen
\item Conozca la herramienta C\+Make y sepa usarla para construir sus programas ejecutables
\end{DoxyItemize}

\subsubsection*{Rúbrica de evaluacion de esta práctica}

Se señalan a continuación los aspectos más relevantes (la lista no es exhaustiva) que se tendrán en cuenta a la hora de evaluar esta práctica\+:
\begin{DoxyItemize}
\item El alumnado ha de acreditar conocer los conceptos expuestos en el material de referencia de esta práctica.
\item El alumnado ha de acreditar que ha realizado todos los ejercicios propuestos, así como ser capaz de desarrollar otros similares.
\item Ha de acreditar que es capaz de escribir un fichero C\+Make\+Lists.\+txt para automatizar el proceso de compilación de sus programas.
\item El código que escriba ha de estar escrito de acuerdo a los estándares definidos en la \href{https://google.github.io/styleguide/cppguide.html}{\tt Guía de Estilo de Google para C++}.
\item Todos los identificadores que utilice en su programa (constantes, variables, etc.) deberán ser siempre significativos. No utilice nunca identificadores de una única letra, tal vez con la excepción de las variables que utilice para iterar en un bucle.
\item Antes de su ejecución, todos los programas que desarrolle, deben imprimir en pantalla un mensaje indicando la finalidad del programa así como la información que precisará del usuario para su correcta ejecución.
\item Ante la presencia de cualquier bug, el alumnado ha de conocer las técnicas básicas de depuración usando Visual Studio Code.
\item Todos los ficheros de código del proyecto correspondiente a esta práctica han de alojarse en un repositorio de \href{https://github.com/}{\tt Git\+Hub}.
\item Los programas deben contener comentarios adecuados en el formato requerido por \href{https://www.doxygen.nl/index.html}{\tt Doxygen}.
\item Los programas deben estructurarse en directorios diferentes para cada \char`\"{}proyecto\char`\"{} y hacer que cada uno de ellos contenga un fichero {\ttfamily C\+Make\+Lists.\+txt} con la configuración de despliegue del proyecto.
\end{DoxyItemize}

\subsubsection*{La herramienta {\ttfamily cmake}}

\href{https://es.wikipedia.org/wiki/CMake}{\tt C\+Make} es lo que se conoce como un sistema de metaconstrucción. No se utiliza para construir (generar, {\itshape build} en inglés) el programa ejecutable de una aplicación sino que genera archivos de proyecto nativos para la plataforma de destino. Por ejemplo, C\+Make en Windows producirá una solución para Visual Studio; en Linux producirá un fichero Makefile; en mac\+OS producirá un proyecto para X\+Code y así sucesivamente. Eso es lo que la palabra meta significa\+: C\+Make construye sistemas de construcción ({\itshape builders}). La herramienta make es un sistema de construcción, posiblemente el más ubicuo.

Un proyecto basado en C\+Make siempre contiene un fichero {\ttfamily C\+Make\+Lists.\+txt} que describe cómo se estructura el proyecto, la lista de ficheros de código fuente que se ha de compilar, lo que C\+Make debe generar a partir de él y así sucesivamente. Se trata en definitiva de un fichero de configuración para la herramienta C\+Make. C\+Make leerá las instrucciones de ese fichero y producirá el resultado deseado.

Una característica positiva de C\+Make es el llamado \char`\"{}out-\/of-\/source build\char`\"{}. Cualquier fichero requerido para la construcción final, incluyendo los ejecutables, será almacenado en un directorio de construcción separado (usualmente llamado build/). Esto evita que el directorio de origen que contiene el código fuente se llene de ficheros no deseados y hace que sea fácil volver a empezar\+: sólo hay que eliminar el directorio destino de la compilación (directorio build) y listo.

C\+Make es una herramienta muy potente que admite multitud de opciones. En \href{https://cmake.org/cmake/help/latest/index.html}{\tt la documentación} de la herramienta se pueden estudiar en profundidad estas opciones, pero para la utilización que perseguimos realizar en esta asignatura bastará con que estudie detenidamente \href{https://www.internalpointers.com/post/modern-cmake-beginner-introduction}{\tt este breve tutorial}.

En el directorio raíz del repositorio de esta práctica hallará un subdirectorio {\ttfamily fibonacci\+\_\+sum} con el siguiente contenido\+:


\begin{DoxyCode}
fibonacci\_sum
├── CMakeLists.txt             // Fichero de configuración para CMake
├── doc                        // Documentación
├── fibonacci.Doxyfile         // Fichero de configuración para Doxygen
├── LEE\_ME.txt
└── src                        // Código fuente de la aplicación
    ├── fibonacci\_main.cc
    ├── fibonacci\_sum.cc
    ├── fibonacci\_sum.h
    ├── tools.cc
    └── tools.h
\end{DoxyCode}
 Esa estructura de directorios (a la que se añadirán los directorios {\ttfamily build} y -\/opcionalmente {\ttfamily lib}-\/) es habitual en proyectos de desarrollo de software. En este ejemplo se ha tomado la aplicación {\ttfamily fibonacci\+\_\+sum} que calcula la suma de términos pares de la serie de Fibonacci y se ha fragmentado la aplicación en 5 ficheros de código ({\ttfamily $\ast$.cc} y {\ttfamily $\ast$.h}). El fichero de configuración {\ttfamily C\+Make\+Lists.\+txt} contiene la configuración que se utiliza para el despliegue de la aplicación. Al efecto de ilustrar este proceso, se crea una librería {\ttfamily libtools.\+a} que se aloja en el directorio {\ttfamily lib}. El programa binario ({\ttfamily fibonacci\+\_\+sum}) se construye enlazando esta librería con el resto del código objeto producto de la compilación.

Para construir la aplicación, siga los siguientes pasos (que son los habituales)\+: 
\begin{DoxyCode}
$ cd fibonacci\_sum
$ mkdir build
$ cd build
$ cmake ..
$ make
\end{DoxyCode}


El comando {\ttfamily cmake}, usando el fichero de configuración, creará en el directorio {\ttfamily build} el fichero {\ttfamily Makefile} que utiliza el comando {\ttfamily make} para construir la aplicación, cuyo programa binario {\ttfamily fibonacci\+\_\+sum} se crea asimismo en el directorio {\ttfamily build}.

Experimente con este fichero de configuración entregado, {\ttfamily C\+Make\+Lists.\+txt} para adaptarlo a cada uno de sus propios proyectos (ejercicios de la práctica). No es necesario en principio, que construya librerías propias para sus programas. La construcción de una librería se ha incluído en este ejemplo a efectos de ilustrar ese proceso.

\subsubsection*{Trabajo previo}

Antes de realizar los ejercicios de esta práctica, estudie detenidamente el Capítulo 8 (epígrafes 8.\+1-\/8.\+16) del \href{https://www.learncpp.com/cpp-tutorial/81-welcome-to-object-oriented-programming/}{\tt tutorial learncpp}. Muchos de los ejemplos de ese tutorial son los mismos que se utilizan en las clases teóricas de la asignatura, cuyo material debiera Ud. también estudiar.

\subsubsection*{Entorno de trabajo}

Al realizar los siguientes ejercicios cree dentro de su repositorio de esta práctica un directorio diferente para cada uno de los ejercicios (proyectos) con un contenido similar al que se ha entregado para la aplicación de ejemplo {\ttfamily fibonacci\+\_\+sum}. Tómese como ejemplo el primero de los ejercicios. Ponga a cada uno de esos directorios nombres significativos (fechas, complejos, racionales por ejemplo).

Haga que cada uno de sus programas conste de 3 ficheros\+:
\begin{DoxyItemize}
\item Un fichero {\ttfamily fechas.\+cc} (programa principal) que contendrá la función {\ttfamily main} e incluirá el fichero de cabecera {\ttfamily fecha.\+h}.
\item El fichero {\ttfamily fecha.\+h} que contendrá las declaraciones correspondientes a la clase {\ttfamily Fecha}.
\item El fichero {\ttfamily fecha.\+cc} que contendrá el código (definiciones) correspondientes a la clase {\ttfamily Fecha}.
\item Obviamente si el programa principal ({\ttfamily fechas.\+cc}) utiliza otras clases, debería incluir ({\ttfamily \#include}) los correspondientes ficheros de cabecera. Modifique estos nombres de ficheros para adaptarlos al ejercicio en cuestión.
\end{DoxyItemize}

La compilación del programa correspondiente a cada ejercicio se automatizará con un fichero {\ttfamily C\+Make\+Lists.\+txt} que se utilizará con {\ttfamily cmake}.

Así pues, la estructura de directorios y sus contenidos correspondiente al primero de los ejercicios propuestos sería la siguiente\+: 
\begin{DoxyCode}
fechas
    ├── build           // Directorio inicialmente vacío para alojar el programa ejecutable
    ├── CMakeLists.txt  // Fichero de configuración para cmake
    └── src             // Directorio contenedor del código fuente del ejercicio
        ├── fecha.cc
        ├── fecha.h
        └── fechas.cc
\end{DoxyCode}


Desarrolle cada uno de estos ejercicios de forma incremental, probando cada una de las funciones que va Ud. desarrollando. Utilice el depurador de V\+SC para corregir cualquier tipo de error semántico que se produzca en cualquiera de sus desarrollos.

\subsubsection*{Ejercicios}


\begin{DoxyEnumerate}
\item La clase Fecha.
\end{DoxyEnumerate}

Desarrolle una clase {\ttfamily Fecha} que permita representar y gestionar fechas. Incorpore en la clase los miembros de datos y métodos que considere adecuados para la finalidad que se persigue en este ejercicio. Incluya un método que permita determinar si el año correspondiente a una fecha es un año bisiesto o no. Realice un programa cliente {\ttfamily fechas.\+cc} que tome como parámetro una fecha, un número y un nombre de fichero\+: 
\begin{DoxyCode}
./fechas - Gestión de fechas
Modo de uso: ./fechas dd/mm/aa N fichero\_salida.txt
Pruebe ./fechas --help para más información
\end{DoxyCode}
 El programa deberá imprimir en el fichero de salida (tercer parámetro) las N (segundo parámetro) fechas cronológicamente posteriores a la introducida (primer parámetro) con una separación de un día entre fechas sucesivas.


\begin{DoxyEnumerate}
\item La clase Complejo.
\end{DoxyEnumerate}

Todo \href{https://es.wikipedia.org/wiki/N%C3%BAmero_complejo}{\tt número complejo} puede representarse como la suma de un número real y un número imaginario, de la forma {\ttfamily a + bi} donde el término {\ttfamily a} es la parte real, {\ttfamily b} la parte imaginaria e {\ttfamily i} la \href{https://es.wikipedia.org/wiki/Unidad_imaginaria}{\tt unidad imaginaria}.

En este ejercicio se propone desarrollar una clase {\ttfamily Complejo} que permita operar con números complejos.

Separe el diseño de su clase {\ttfamily Complejo} en dos ficheros, {\ttfamily complejo.\+h} y {\ttfamily complejo.\+cc} conteniendo respectivamente la declaración y la definición de la clase. Siga las indicaciones del tutorial \href{https://www.learncpp.com/cpp-tutorial/89-class-code-and-header-files/}{\tt Class code and header files} para realizar esta separación de su clase en dos ficheros. Siga igualmente las indicaciones del tutorial \href{https://www.learncpp.com/cpp-tutorial/header-guards/}{\tt Header guards} para incluir (siempre de ahora en adelante) {\itshape header guards} (guardas de cabecera) en sus ficheros de definiciones ({\ttfamily $\ast$.h}) de modo que se evite la inclusión múltiple del mismo fichero.

Desarrolle un programa cliente {\ttfamily complejos.\+cc} que permita operar con números complejos y haga uso de la clase {\ttfamily Complejo} que diseñe. La clase {\ttfamily Complejo} ha de contener al menos métodos que permitan ({\itshape print()}) imprimir un número complejo así como sumar ({\itshape Add()}) y restar ({\itshape Sub()}) números complejos. Así la función {\itshape main} del programa {\ttfamily complejos.\+cc} podría contener sentencias como las siguientes\+:


\begin{DoxyCode}
main() \{
  Complejo complejo1\{4, 5\}, complejo2\{7, -8\};
  Complejo resultado;
  resultado = add(complejo1, complejo2);
  resultado.print();
  resultado = sub(complejo1, complejo2);
  resultado.print();
\}
\end{DoxyCode}
 que imprimirían en pantalla los resultados de la suma y referencia de números complejos indicada.

Incluya (discrecionalmente) cualesquiera otras operaciones que considere adecuadas como métodos en la clase {\ttfamily Complejo}.


\begin{DoxyEnumerate}
\item La clase Racional.
\end{DoxyEnumerate}

Un \href{https://en.wikipedia.org/wiki/Rational_number}{\tt número racional} tiene un numerador y un denominador de la forma {\ttfamily p/q} donde {\ttfamily p} es el numerador y {\ttfamily q} el denominador. Por ejemplo, 1/3, 3/4 y 10/4 son números racionales.

Un número racional no puede tener denominador 0, pero sí puede ser cero el numerador. Todo número entero {\ttfamily n} es equivalente al racional {\ttfamily n/1}. Los números racionales se utilizan en cálculos precisos que involucran fracciones. Por ejemplo, {\ttfamily 1/3 = 0.\+33333 ...}. Este número no puede ser representado de forma precisa en formato de punto flotante utilizando los tipos float o double. Para obtener resultados precisos es conveniente usar números racionales.

C++ dispone de tipos de datos para enteros y números en punto flotante, pero no para racionales. En este ejercicio se propone el diseño de una clase para representar números racionales.

Desarrolle un programa cliente {\ttfamily racionales.\+cc} que permita operar con números racionales y haga uso de la clase {\ttfamily Racional} que ha de diseñarse.

Las siguientes deben tomarse como especificaciones del programa a desarrollar\+:
\begin{DoxyItemize}
\item Separe el diseño de su clase {\ttfamily Racional} en dos ficheros, {\ttfamily racional.\+h} y {\ttfamily racional.\+cc} conteniendo respectivamente la declaración y la definición de la clase.
\item La clase {\ttfamily Racional} incluirá al menos métodos para\+:
\begin{DoxyItemize}
\item Crear objetos de tipo {\ttfamily Racional}. Se debe implementar un constructor por defecto y uno parametrizado.
\item Escribir (a fichero o a pantalla) un objeto de tipo {\ttfamily Racional}.
\item Leer (por teclado o desde fichero) un objeto de tipo {\ttfamily Racional}.
\item Sumar dos objetos de tipo {\ttfamily Racional}.
\item Restar dos objetos de tipo {\ttfamily Racional}.
\item Multiplicar dos objetos de tipo {\ttfamily Racional}.
\item Dividir dos objetos de tipo {\ttfamily Racional}.
\item Comparar objetos de tipo {\ttfamily Racional}.
\end{DoxyItemize}
\item El programa ha de permitir leer un fichero de texto en el que figuran pares de números racionales separados por espacios de la forma\+:
\end{DoxyItemize}


\begin{DoxyCode}
a/b c/d
e/f g/h
  ...
\end{DoxyCode}


y para cada par de números racionales, el programa ha de imprimir en otro fichero de salida todas las operaciones posibles con cada uno de los pares de números del fichero de entrada, de la forma\+:


\begin{DoxyCode}
a/b + c/d = n/m
  ...
\end{DoxyCode}


Si el programa se ejecuta sin pasar parámetros en la línea de comandos, debemos obtener el siguiente mensaje\+:


\begin{DoxyCode}
./racionales -- Números Racionales
Modo de uso: ./racionales fichero\_entrada fichero\_salida
Pruebe ./racionales --help para más información
\end{DoxyCode}


Si el programa se ejecuta pasando como parámetro la opción {\ttfamily -\/-\/help} se ha de obtener\+:

``` ./racionales -- Números Racionales Modo de uso\+: ./racionales fichero\+\_\+entrada fichero\+\_\+salida

fichero\+\_\+entrada\+: Fichero de texto conteniendo líneas con un par de números racionales\+: {\ttfamily a/b c/d} separados por un espacio fichero\+\_\+salida\+: Fichero de texto que contendrá líneas con las operaciones realizadas\+: {\ttfamily a/b + c/d = n/m} ``` \subsubsection*{Referencias}


\begin{DoxyItemize}
\item \href{https://es.wikipedia.org/wiki/CMake}{\tt C\+Make}
\item \href{https://www.internalpointers.com/post/modern-cmake-beginner-introduction}{\tt Introduction to modern C\+Make for beginners}
\item \href{https://es.wikipedia.org/wiki/N%C3%BAmero_complejo}{\tt Números complejos}
\item \href{https://en.wikipedia.org/wiki/Rational_number}{\tt Rational Number}
\item \href{https://www.learncpp.com/cpp-tutorial/89-class-code-and-header-files/}{\tt Class code and header files}
\item \href{https://www.learncpp.com/cpp-tutorial/header-guards/}{\tt Header guards}
\item \href{https://en.wikipedia.org/wiki/Doxygen}{\tt Doxygen}
\item \href{https://developer.lsst.io/cpp/api-docs.html}{\tt Documenting C++ Code}
\item \href{https://google.github.io/styleguide/cppguide.html}{\tt Google C++ Style Guide} 
\end{DoxyItemize}